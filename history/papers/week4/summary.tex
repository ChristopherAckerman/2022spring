\documentclass[dvipsnames]{article}
\usepackage{amsmath,amsthm,amssymb}
\usepackage{graphicx}
\usepackage{hyperref}
\usepackage{textcomp}
% \usepackage{dsfont}
\usepackage{tabularx}
\usepackage{tikz}
\usepackage{physics}
\usepackage{changepage}% http://ctan.org/pkg/changepage
\usetikzlibrary{scopes,calc,arrows}
\usepackage{setspace}
\usepackage[makeroom]{cancel}
\usepackage{enumitem}
\usepackage[margin=1in]{geometry}
\usepackage[T1]{fontenc}
\usepackage[utf8]{inputenc}
\usepackage{tabularx,ragged2e,booktabs,caption}
\usepackage{wrapfig,lipsum,booktabs}
\usepackage{hanging}
\usepackage{multicol}
\usepackage{multirow}
\usepackage{blindtext}
\usepackage{booktabs}
\usepackage{color}
\usepackage{dcolumn}
% \usepackage{minted}
% \definecolor{light}{rgb}{0.35, 0.35, 0.35}
% \def\light#1{{\color{light}#1}}
\usepackage{mathtools}
\newcommand{\p}{\mathbb{P}}
\newcommand{\E}{\mathbb{E}}
\newcommand{\R}{\mathbb{R}}
\newcommand{\Var}{\operatorname{Var}}
\newcommand{\gr}{\textcolor{ForestGreen}}
\newcommand{\rd}{\textcolor{red}}



\title{Summary for Acemoglu, Autor and Lyle (2004) and Farber, Herbst, Kuziemko and Naidu (2021)}

\begin{document}
\maketitle
\section{Acemoglu, Autor and Lyle (2004)}
% \subsection{Question}
This paper investigates the impact of women entering the labor market on the wage structure. 
% \subsection{Motivation}
This paper addresses how the market viewed/utilized both male and female labor in the 20th century. It investigates whether men and women are substitutes in firms' production functions, what types of female workers compete with what types of male workers, and how this impacts wages. This is a natural question as female labor force participation has increased over time.

% \subsection{What it Does}
The authors rely on variation induced by WWII mobilization for their research design. During WWII, men were sent to fight in large numbers, inducing women to enter the labor force. However, variation in mobilization rates across states created exogenous variation in the demand for female labor in these states. 


% \subsection{Contribution}
This paper provides insight on how compensation changes as women enter the market, and on how firms view male- and female labor. First, women entering the labor market reduces wages for both men and women. However, it reduces wages \emph{more} for women, indicating a labor demand elasticity between $-1.2$ and $-1.5$. This impact is not constant across all \emph{types} of jobs: women are very close substitutes for average male workers, but firms do not view women as good substitutes for either very high- or low-skilled male workers.

\section{Farber, Herbst, Kuziemko and Naidu (2021)}
% \subsection{Question}
This paper investigates the relationship between unions and inequality. 
% \subsection{Motivation}
There is existing research on the impact of unions on wages and labor market outcomes, but it has conflicting findings and is limited by data availability concerns. The authors aim to both improve on existing data and provide evidence about the \emph{causal} impact of union concentration.


% \subsection{What it Does}
CPS data does not include union membership before 1973, but the authors collect data from 1936--1986 by combining data from over 500 surveys. This is similar to the approach in J\'acome, Kuziemko and Naidu (2021). The authors then use their data to ``horse race'' the institutional and market forces models of unions' impact on wages. Finally, they use the ``Wagner Shocks'' and ``War-spending shock'' to generate an IV and identify the causal effect of union concentration. 


% \subsection{Contribution}
The paper's first main contribution is some interesting descriptive facts about unions' racial composition in the 20th century. They have three main findings First, after conditioning on state of residence, Blacks were not underrepresented in unions. Second, Black over-representation (outside of the South) is largely due to the focus on organized lower-skilled jobs. Third, membership is an incomplete measure of union experience: even though Blacks were in unions, they were often discriminated against in these unions.


Their second contribution is an interesting non-finding. The expand on Card (2001), showing that not only was the union wage premium stable over more than 20 years, but in fact it was stable for more than 90 years. This is surprising since they also document dramatic swings in union membership, and it is hard to reconcile a stable wage premium with a bargaining model that allows large fluctuations in membership. The estimates of a sizable union premium contrast with recent findings, and point to future question in this line of research.


Finally, the authors try to differentiate between two models of unions' effects on wages: do unions affect the wages of union workers only, or of all workers? Looking at the impact of unions on inequality, they find that union membership decreased the Gini coefficient by about 0.025 at peak union density, and they find that the within-union effect dominates the union-nonunion difference. Depending on the measure of inequality, the authors find that changes in union concentration explain between 18--80\% of the changes in inequality between 1936 and 1986. The authors find a larger impact of unions on wages based on time-series data, thus indicating that unions have spillover effects onto non-union workers. 
\end{document}
%%% Local Variables:
%%% mode: latex
%%% TeX-master: t
%%% End:
