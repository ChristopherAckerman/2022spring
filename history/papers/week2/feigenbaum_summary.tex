\documentclass[dvipsnames]{article}
\usepackage{amsmath,amsthm,amssymb}
\usepackage{graphicx}
\usepackage{hyperref}
\usepackage{textcomp}
% \usepackage{dsfont}
\usepackage{tabularx}
\usepackage{tikz}
\usepackage{physics}
\usepackage{changepage}% http://ctan.org/pkg/changepage
\usetikzlibrary{scopes,calc,arrows}
\usepackage{setspace}
\usepackage[makeroom]{cancel}
\usepackage{enumitem}
\usepackage[margin=1in]{geometry}
\usepackage[T1]{fontenc}
\usepackage[utf8]{inputenc}
\usepackage{tabularx,ragged2e,booktabs,caption}
\usepackage{wrapfig,lipsum,booktabs}
\usepackage{hanging}
\usepackage{multicol}
\usepackage{multirow}
\usepackage{blindtext}
\usepackage{booktabs}
\usepackage{color}
\usepackage{dcolumn}
% \usepackage{minted}
% \definecolor{light}{rgb}{0.35, 0.35, 0.35}
% \def\light#1{{\color{light}#1}}
\usepackage{mathtools}
\newcommand{\p}{\mathbb{P}}
\newcommand{\E}{\mathbb{E}}
\newcommand{\R}{\mathbb{R}}
\newcommand{\Var}{\operatorname{Var}}
\newcommand{\gr}{\textcolor{ForestGreen}}
\newcommand{\rd}{\textcolor{red}}


\title{Summary of Feigenbaum (2018)}
\author{Christopher Ackerman}
\date{\today}
\begin{document}
\maketitle
This paper compares different measures of intergenerational mobility. It looks at a standard dataset, Iowa from 1915--1940, and examines how different definitions of intergenerational mobility compare. Because Iowa was not representative of the US as a whole, it is important to consider compositional effects when comparing Feigenbaum's mobility estimates to the rest of the US. The four measures of mobility that the author uses are income mobility, education mobility, occupational mobility, and name-based mobility.

All of these mobility measures paint a similar picture. Iowa, at least from 1915--1940, featured high mobility---much higher mobility than modern day Iowa. But mobility was not the same for \emph{all} Iowans. Rural residents, and individuals with foreign-born grandparents were more mobile than their peers. The author then compares these historical estimates to estimates of modern-day mobility in Iowa. Mobility rates for the older cohort are higher than for men born since 1960. There are a number of challenges when comparing historical estimates across time periods, but the stability and persistence of this finding across different mobility measures adds more evidence that mobility has been declining in the US.

I liked this paper a lot, and found it very helpful---for me. I'm not sure how helpful it is for practicing economic historians. As somebody who has never been exposed to historical data or mobility data before, this was a great primer. But it reads much more like a technical appendix or an \emph{Annual Reviews} article than a normal economics paper. The results in this paper seem like a fairly marginal contribution, so I'm not sure how much value this piece has except as a teaching tool for new economic historians.
\end{document}
%%% Local Variables:
%%% mode: latex
%%% TeX-master: t
%%% End:
