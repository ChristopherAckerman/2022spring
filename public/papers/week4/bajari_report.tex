\documentclass[dvipsnames]{article}
\input{../../../article_preamble.tex}
\title{Report on Bajari, Houghton and Tadelis (2014)}
\author{Christopher Ackerman}
\date{\today}
\begin{document}
\maketitle
This paper looks at highway procurement auctions, extending the important work by Porter and Zona. This setup is similar, but adds a strategic element to the \emph{type} of bid that a firm submits. Firms are selected based on their overall cost estimate, but are actually paid based on realized costs. If firms think that a certain part of the product (e.g. sidewalks) is likely to have an over-run, they have an incentive to quote a higher price for that item so that they will receive greater compensation during the over-run. In order to more closely match the data, the authors impose a penalty for highly irregular bids (otherwise the optimal strategy is to submit a winning bid with certain items estimated at zero cost).

The authors start with a (compelling) reduced form analysis of their auction data. Their initial finding is that a contract expecting a 10 percent overrun on some item would shade his bit up by approximately 0.5 percent, which is economically modest. The authors are careful when they specify their reduced-form regressions, ensuring that their estimating equations correspond to the fairly complicated FOCs they derive (see for example equations 4 and 7). The reduced form analysis yields a surprising result: firms actually find over-runs to be costly. Ex ante, I thought that firms would shade their bids down and then reap additional profits from cost overruns. However, these results indicate that firms find over-runs ungainly, unprofitable, and not worth their time. Thus, if they expect significant over-runs from a project, they shade their bids \emph{up} in anticipation of the additional headaches.

In order to compare the specific frictions (adjustments) that the authors are focusing on in this paper, they then estimate a structural model so that they can compare the relative importance of adjustment costs, market power, and strategic bidding. To estimate firm-specific bid distributions, the authors have to impose a semi-parametric restriction to avoid dimensionality issues. On balance this doesn't seem like a huge problem. To estimate adaptation costs, the authors again impose a multiplicative structure similar to Krasnokutskaya (2011). They go to decent lengths to motivate this assumption, and they also use the structural model largely as a way to show the robustness of their reduced form findings, so this again does not seem like a major weakness. These estimates provide enough information to evaluate each firms' markup.

The estimated adjustment costs are surprisingly large and go in the opposite direction of what I expected ahead of time. The authors go to great lengths to highlight the importance of these costs, but I think they're over-selling the actual empirical relevance a bit. Yes, this behavior is interesting and worth highlighting, but the result (in this market at least) is that firms are charging small markups anyway and competition seems to be pretty strong. The policy recommendation they make is heavily qualified---while cost-plus contracts get around some of the complicated adjustment-cost issues, they are much more open to manipulation and bid rigging, which seems to be a more relevant concern in this market. I would be curious to see if these adjustment costs can be estimated for different markets, and whether there are meaningful differences across markets. For markets where cartels or other anti-competitive behavior are less likely, it may be worth more seriously considering different procurement mechanisms. This type of cross-market comparison is particularly salient because Chris Saw presented a paper in Michael Reubens' class yesterday where it seemed that the results were \emph{heavily} dependent on the market (craft beer) chosen, and that the authors would be unlikely to get any meaningful results in different applications.
\end{document}
%%% Local Variables:
%%% mode: latex
%%% TeX-master: t
%%% End:
